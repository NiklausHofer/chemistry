\documentclass[11pt,paper=a4,final]{scrartcl}
\usepackage[utf8]{inputenc}
\usepackage{geometry}           %allows us to specify the 'seitenrand'
\usepackage{graphicx}           %package used to include graphics
\usepackage{hyperref}           %used to make klickable links
\usepackage{listings}
%\usepackage{tabularx}
%\usepackage{pdflscape}
\usepackage[figuresright]{rotating}
\usepackage{nameref}
\usepackage{longtable}
\usepackage{enumitem}

\hypersetup{
    colorlinks,
    citecolor=black,
    filecolor=black,
    linkcolor=black,
    urlcolor=black
}

\usepackage{fancyhdr}
\pagestyle{fancy}
% \setlength{\parskip}{0pt}
% \setlength{\baselineskip}{0pt}
%\parindent 0pt 
%\parskip 11pt
%\parsep 0pt 
%\itemsep 0pt 
%\topsep 0pt 

\geometry{a4paper, top=20mm, right=20mm, bottom=20mm, left=20mm}

%defining header and footer
\fancyhf{}      %delete default values
\setlength{\headwidth}{\textwidth}      %header and footer width equal the text width
%\fancyhead[LE,LO]{\includegraphics[scale=0.6]{header.png}}
\fancyhead[LE,LO]{Niklaus Hofer}
\fancyhead[RE,RO]{Notizen zum Chemiebuch}
\fancyfoot[CE,CO]{Speicherdatum: \today{}}
\fancyfoot[RE,RO]{\thepage}

%New page for every section
%\let\stdsection\section
%\renewcommand\section{\newpage\stdsection}

\title{Notizen zum Chemiebuch}
\author{Niklaus Hofer}
\date{\today{}}

\begin{document}
\maketitle
% No new page after the title for this one

\begin{itemize}
  \item Merkmale von Reinen Stoffen und Gemischen: Seite 11, unten
  \item Beispiele von Mischungen: Seite 12, unten
  \item Zusammensetzung der Luft: Seite 13, oben
  \item reine Stoffe und Verbindungen: Seite 14/15
  \item Beschreibung der Agregatszust\"ande: Seite 16, unten
  \item Erkl\"arung des Aufbaus des Periodensystems: Seite 18, oben
  \item Trennmethoden: Seite 24 bis 30
    \begin{itemize}
      \item Destillation: 25
      \item Chromatografie: 26
      \item Filtration, Adsorption, Extraktion: 28
      \item Sedimentieren, Zentrifugieren: 30
    \end{itemize}
\end{itemize}

Bei den reinen Stoffen wird unterschieden zwischen Verbindungen und elementaren
Stoffen. Verbindungen, wie Wasser, lassen sich durch die Analyse zerlegen.
elementare Stoffe hingegen, k\"onnen nicht weiter zerlegt werden. Der Prozess
des Zusammensetzens von elementaren Stoffen zu Verbindungen wird Synthese
genannt. 

Der direkte \"Ubergang vom festen in den gasf\"ormigen Zustand wird
"sublimieren" genannt.

Das Gesetz der konstanten Proportionen beschreibt, dass jeweils dieselbe Menge
verschiedener Stoffe zusammen reagieren. Seite 36 und 37. Hier finden sich auch
Erlaeuterungen zu Stickstoff-Sauerstoff-Verbindungen.

Gesetz der multiplen Proportionen: Seite 38

Columbo-Gesetz zur elektrischen Ladung und Anziehung von Koerpern: Seite 39.
Weiter auf Seite 39 wird in diesem Zusammenhang auch die Elementarladung
erklaert.

Kathode: negativer Pol - Anode: positiver Pol

Aufbau von Atomen aus den Elektronen, Neutronen und Protonen wird auf Seite 43
zusammengefasst.

Beschreibung der Elementenschreibeweise\\: \(^{20}_{10}Ne\) oder \(Ne-20: 10p^+, 10e^-, 10n\)\\
                                         \(^{22}_{10}Ne\) oder \(Ne-22: 10p^+, 10e^-, 12n\)\\
					 (Seite 44)

Die Begriffe Isotope und Neuclide wie auch Neutronenzahl, Nucleonenzahl,
Protononenzahl, sind auf Seite 44 unten erklaert.

Die atomare Masse u (fuer unit) und deren genauer Wert werden auf Seite 48
beschrieben.

Mol werden auf Seiten 49 und 50 erklaert. Ein Mol enth\"alt \(6.02 \cdot
10^{23}\) Teilchen. Diese Zahl nennt man die Avogadro Konstante.

In welchem Verhaeltnis reagieren zwei Stoffe? Wie das berechnet werden kann, ist
auf Seite 51 beschrieben.

Auf Seite 52, in der Mitte, befindet sich eine Tabelle mit verschiedenen
Groessen, Eihneiten und Zeichen.

Lesen des Periodensystems: Nucleonenzahl entspricht der relativen Atommasse.
Anzahl Protonen ist gleich der Anzahl Elektronen, entspricht der Ordnungszahl.
Anzahl Neutronen ist Nucleonenzahl weniger der Anzahl Protonen (Ordnungszahl).
Die Masse in g eines Atoms laesst sich berechnen, indem die relative Atommasse
mit \(1.661 \cdot 10^{-24} \) multipliziert wird.\\ Wie diese Werte zu berechnen
sind, ist auf Seite 53 anhand eines Beispiels beschrieben.

Aufbau des Atomkerns: Seite 66

Die maximale Elektronenzahl der Schalen eines Atomes wird mit der Formel \( 2
\cdot x^2 \) berechnet. x steht fuer die Schalennummer (s.S. 68).

Auf Seite 69 ist der Aufbau des Periodensystems nach Anzahl Schalen und
Aussen(Valenz)elektronen beschrieben.

Der Unterschied zwischen Metallen und Nichtmetallen ist auf Seite 70 erklaert.

Chemische Vorgaenge spielen sich meist in der Valenzschale ab. Die Wolken
beschreiben Aufenthaltsorte von Elektronen. Die Schalen haben unterschiedlich
viele Wolken. Jede Wolke enthaelt max. 2 Elektronen. Wolken gleicher Energie
werden zuerst alle einfach belegt, bevor die Doppelbelegung beginnt. (s.S. 71)
Wie dies symbolisch dargestellt wird, ist auf Seite 72 erlaeutert.

Innerhalb der Schalen existieren Unterschalen. Bei der Verteilung der Elektronen
haben diese Vorrang (s.S. 72, unten). In welcher Reihenfolge die Unterschalen
genau belegt werden, stellt die Abbildung auf Seite 74 dar. Wie die Unterschalen
im Periodensystem dargestellt werden, ist auf Seite 76 erklaert.

Rumpf, Rumpfladung, Metalle: Seite 76, unten

Die Wellenfunktionen s, p, d und f werden auf Seite 87 beschrieben.

Bei chemischen Reaktionen entsteht aus den Edukten ein Produkt. Dabei findet ein
Waermeumsatz statt. Eine Reaktion ist umkehrbar. Reaktionen die mehr Energie
abgeben als dafuer aufgewendet werden muss nennt man exotherm, das Gegenteil
endotherm. Genaueres dazu (Ausdruecke) findet sich auf Seite 94. Seite 103
erkl\"art wie die Elemente beschriftet werden. Seite 104 oben erlaeutert die
Normbedingungen.

Bei Normbedingungen (\(0^{\circ}C\) 273.15K, 101325Pa) hat 1Mol Gas ein Volument
von 22.4L. Formel zur Berechnung des Zusammenhangs auf Seite 99, oben.

Regeln zum Niederschreiben der Reaktionsgleichungen (insbesondere der
Mengenschereibweise), finden sich auf Seite 103, unten.
\begin{table}[h!]
  \centering
  \begin{tabular}{|l|l|p{6cm}|}\hline
    Metall + Nichtmetall & Ionenbindung / Salze & \(Me^+, NMe^- \) Metall gibt
    Elektronen an das Nichtmetall ab\\ \hline
    Nichtmetall + Nichtmetall & Atombindung/ Molek\"ul & Verbindung der Wolken
    \\ \hline
    Metall + Metall & Metallbindung / Legierung & xMet1 + yMet2 \\ \hline
  \end{tabular}
  \caption{}
  \label{tab:}
\end{table}
Ionen sind elektrisch geladene Teilchen. (Elektronen sind negativ geladen)

Als Ionisierungsenergie wird die Energie beschrieben, die benoetigt wird um
einem Atom Elektronen abzuspalten.

Endungen: -O -oxid -S -sulfid -N -nitrid -H -hybrid -C -carbid

Anzahl aussenelektronen laesst sich ermitteln, indem man die Zahlen deer kleien
Kaestchen addiert. Bei der Ionenbindung wird die rote Zahl dann wieder
substrahiert.
Merkmale von Salzen: nur leitf\"ahig wenn gel\"ost, hohe temp, spr\"ode.

Teilchenverh\"altnis anhand von Reaktion ermitteln: Gewicht des Produktes -
Gewicht Edukt1 = Gewicht Edukt2.
\(\frac{masse}{\frac{masse}{mol}} = mol \)
\end{document}
