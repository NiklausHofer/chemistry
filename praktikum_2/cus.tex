\documentclass[11pt,paper=a4,final]{scrartcl}
\usepackage[utf8]{inputenc}
\usepackage{geometry}           %allows us to specify the 'seitenrand'
\usepackage{graphicx}           %package used to include graphics
\usepackage{hyperref}           %used to make klickable links
\usepackage{listings}
\usepackage{tabularx}
\usepackage{pdflscape}
\usepackage[figuresright]{rotating}
\usepackage{nameref}
\usepackage{longtable}
\usepackage{enumitem}
\usepackage{ngerman} % Make the document German
% Make the document German
\usepackage{ngerman}
\usepackage{fancyhdr}
\usepackage{lipsum}
\usepackage{mdwlist}
\usepackage{textcomp}

\hypersetup{
    colorlinks,
    citecolor=black,
    filecolor=black,
    linkcolor=black,
    urlcolor=black
}


\title{Kupfersulfid herstellen}
\author{Roland Rytz \and Niklaus Hofer}
\date{\today{}}

% Make title and author accessible in the header/footer
\makeatletter
  \let\Title\@title
  \let\Author\@author
\makeatother

\pagestyle{fancy}

%\geometry{a4paper, top=20mm, right=20mm, bottom=20mm, left=20mm}

\fancyhf{}      %delete default values
\setlength{\headwidth}{\textwidth}      %header and footer width equal the text width
\lhead{\Author}
\rhead{\Title}
\fancyfoot[CE,CO]{Speicherdatum: \today{}}
\fancyfoot[RE,RO]{\thepage}

\begin{document}
\maketitle
\newpage
\( 1mol Schwefel = 32.06g \)\\
\( 0.03 \cdot 32.06g = 0.9618g \)
\section{Messwerte, Beobachtungen}
\subsection{Genauigkeitsabsch\"atzung}
Die Werte wurden mit einer Mettler Toledo B303-S ermittelt.\\
Genauigkeit der Wage: \(\pm 0.002g\)\\
Der Kupfersulfidstreifen des ersten Versuches ist zu Boden gefallen. \\
Genauigkeit des ersten Kupfersulfidstreifens durch Fallenlassen: \(\pm 10\% \)
\subsection{Messwerte}
{
\begin{table}[h!]
\centering
  \begin{tabular}{|l|l|l|l|l|l|}
    \hline
    \bf Versuch			& \bf Ungenauigkeit	& \bf Nr. 1	& \bf Nr. 2	& \bf Nr. 3	& \bf Nr. 4	\\ \hline
    Masse Schwefel (S) 		& \( \pm 0.002g \)	& 0.965g	&	0.960g & 0.965g & 1.097g \\ \hline
    Masse Kupfer (Cu)		& \( \pm 0.002g \)	& 0.286g	&	0.176g & 0.345g & 0.215g \\ \hline
    Masse Kupfersulfid (CuS)	& \( \pm 0.002g \)	& 0.302g \(+ ~ 10\% \)       & 	0.222g & 0.434g & 0.273g \\ \hline
  \end{tabular}
  \caption{Tabelle mit den Messwerten aus den vier Versuchen}
  \label{tab:messwerte}
\end{table}
\subsection{Beobachtungen}
Die ersten beiden Werkst\"ucke sind uns leider jeweils zu Boden gefallen. Das
erste davon ist dabei zerbrochen. Zwar hat Niklaus, der es fallen gelassen hat,
versucht, die Bruchst\"ucke wieder einzusammeln und auch zu w\"agen. Dabei ging
aber mindestens eines der St\"ucke verloren. Dieses betr\"agt sch\"atzungsweise
\( \frac{1}{12} \) der Masse.

Das zweite St\"uck ist ebenfalls zu Boden gefallen. Nach unseren Beobachtungen
ging dabei aber kein Material verloren.

Beim w\"agen der letzten Ladung Schwefel, hatte Niklaus Probleme mit der Wage.
Diese hat immer wieder andere Werte angezeigt. Wie sich herausgestellt hat, war
die Wage nach dem S\"aubern nicht richtig zusammen gesetzt worden. Da die genaue
Menge Schewefel aber nicht so wichtig ist und wir den Missstand rechtzeitig
bemerkt haben, sollte das keine Rolle spielen.

Beim Erhitzen des Schwefels ist dieses braun und fl\"ussig geworden.
Anschliessen ist ein hellgelber Dampf aufgestiegen. Dieser ist im Reagenzglas
langsam gestiegen. Wenn dann das Kupfer erhitzt wurde, so hat sich dieses erst
schwarz verf\"arbt. In einigen F\"allen, ist das sehr schlagartig von Statten
gegangen. Nach weiteren Erhitzen ist ein Glutstreiffen von unten nach
Oben \"uber das Kupfer gewandert. Es hat ausgesehen als sei etwas verbrannt.
Zur\"uck geblieben ist das Kupfersulvid, eine sehr spr\"ode, graue Masse.

Beim weiteren Erhitzen des Produktes in einem sauberen Reagenzglas, sind
weitere R\"uckst\"ande verdampft und das Werkst\"uck so ges\"aubert. Seine
dunkelgraue Farbe war nun besser sichtbar.
\subsection{Interpretation}
Die eigentliche Reaktion der Edukte, Schwefel und Kupfer, zu Kupfersulvid hat
wohl im Punkt des Verbrennens stattgefunden. ge\"aussert hat sich dies \"uber
den Glutstreiffen, der \"uber das Kupfer gewandert ist.  Doch bereits vor diesem
Schritt hat die oberste Schicht des Kupfers mit den Schwefeld\"ampfen reagiert.
Dies manifestierte sich durch das schwarzwerden des Materials.
\subsection{Beobachtungen}
\section{Auswertung, Interpretation}
\bibliography{}{}
\bibliographystyle{plain}
\end{document}

