\documentclass[11pt,paper=a4,final]{scrartcl}
\usepackage[utf8]{inputenc}
\usepackage{geometry}           %allows us to specify the 'seitenrand'
\usepackage{graphicx}           %package used to include graphics
\usepackage{hyperref}           %used to make klickable links
\usepackage{listings}
\usepackage{tabularx}
\usepackage{pdflscape}
\usepackage[figuresright]{rotating}
\usepackage{nameref}
\usepackage{longtable}
\usepackage{enumitem}
\usepackage{footnote}
% Make the document German
\usepackage{ngerman}
% allow rowspan
\usepackage{multirow}

\hypersetup{
    colorlinks,
    citecolor=black,
    filecolor=black,
    linkcolor=black,
    urlcolor=black
}

\usepackage{fancyhdr}
\pagestyle{fancy}
% \setlength{\parskip}{0pt}
% \setlength{\baselineskip}{0pt}
%\parindent 0pt 
%\parskip 11pt
%\parsep 0pt 
%\itemsep 0pt 
%\topsep 0pt 

\geometry{a4paper, top=20mm, right=20mm, bottom=20mm, left=20mm}

%defining header and footer
\fancyhf{}      %delete default values
\setlength{\headwidth}{\textwidth}      %header and footer width equal the text width
%\fancyhead[LE,LO]{\includegraphics[scale=0.6]{header.png}}
\fancyhead[LE,LO]{Roland Peka Rytz, Niklaus Manuel Hofer}
\fancyhead[RE,RO]{Chromatographie}
\fancyfoot[CE,CO]{Speicherdatum: \today{}}
\fancyfoot[RE,RO]{\thepage}

%New page for every section
%\let\stdsection\section
%\renewcommand\section{\newpage\stdsection}

\title{Chromatographie}
\author{Roland Peka Rytz, Niklaus Manuel Hofer}
\date{\today{}}

\begin{document}
\maketitle

\section{Messwerte, Beobachtungen}
\subsection{Messwerte}
Genauigkeit der Chromatographie-Plaaten: unbekannt\\
Genauigkeit der Messung (siehe unten): \( \pm 0.3cm\) \\
Ungenauigkeit beim Vermessen der L\"osungsmittel-Front: vernachl\"assigbar \\
Die Temepratur im Arbeitszimmer betrug: \(20^\circ C \pm 3^\circ C\)

\begin{savenotes} %Handles footnotes within tables
  \begin{table}[ht]
    \centering
    \begin{tabular}{|l|l|l|l|l|}
      \hline
      \bf Farbe Nr.	& \bf Teilfarbex	& \bf L\"osungsmittel-Front	& \bf Distanz		& \bf Rf		\\ \hline
       1 (hellblau)	& hellblau		& \multirow{6}{*}{6.1 cm }
										& 4.0 cm \(\pm 3 mm \)	& 0.607 - 0.705		\\ \cline{1-2} \cline{4-5}
      2 (dunkelgr\"un)	& dunkelgr\"un		&				& 5.8 cm \(\pm 3 mm \)	& 0.902 - 1.000		\\ \cline{1-2} \cline{4-5}
      \multirow{2}{*}{3 (rot) }
			& pink			&				& 4.6 cm \(\pm 3 mm \)	& 0.705 - 0.803		\\ 
			& orange		&				& 6.1 cm 		& 1.00 (nicht definiert)\\ \cline{1-2} \cline{4-5}
      \multirow{2}{*}{4 (hellgr\"un) }
 			& hellblau		&				& 3.3 cm \(\pm 3 mm \)	& 0.500 - 0.600		\\
			& gelb			&				& 6.1 cm		& 1.00 (nicht definiert)\\ \hline
      5 (gelb)		& orange/dunkelgebl	& \multirow{6}{*}{6.1 cm }
										& 6.1 cm		& 1.00 (nicht definiert)\\ \cline{1-2} \cline{4-5}
      \multirow{4}{*}{6 (braun)}
			& Violett		&				& 0.7 cm \(\pm 3 mm \)	& 0.066 - 0.164		\\
			& hellblau		&				& 2.7 cm \(\pm 3 mm \)	& 0.393 - 0.492		\\
      			& pink			&				& 3.9 cm \(\pm 3 mm \)	& 0.590 - 0.689		\\
      			& Orange		&				& 6.1 cm 		& 1.00 (nicht definiert)\\ \cline{1-2} \cline{4-5}
      \multirow{2}{*}{7 (schwarz}
      			& gelb			&				& 4.0 cm \(\pm 6 mm \)	& 0.557 - 0.754		\\
      			& schwarz		&				& 6.1 cm		& 1.00 (nicht definiert)\\ \cline{1-2} \cline{4-5}
      	
    \end{tabular}
    \caption{Rf-Werte}
    \end{table}
\end{savenotes}

\section{Fehlerabsch\"atzung}
\begin{itemize}
  \item Die Genauigkeit der Chromatographie-Platten ist leider unbekannt.
  \item Die Messung der \( R_X \) Werte ist aber ohnehin nicht besonders genau. Auch, da nicht genau klar ist wo diese gemessen werden. Einige Farbklekse laufen nach oben langsam aus.
\end{itemize}
\end{document}
