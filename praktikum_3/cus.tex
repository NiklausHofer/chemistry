\documentclass[11pt,paper=a4,final]{scrartcl}
\usepackage[utf8]{inputenc}
\usepackage{geometry}           %allows us to specify the 'seitenrand'
\usepackage{graphicx}           %package used to include graphics
\usepackage{hyperref}           %used to make klickable links
\usepackage{listings}
\usepackage{tabularx}
\usepackage{pdflscape}
\usepackage[figuresright]{rotating}
\usepackage{nameref}
\usepackage{longtable}
\usepackage{enumitem}
\usepackage{ngerman} % Make the document German
% Make the document German
\usepackage{ngerman}
\usepackage{fancyhdr}
\usepackage{lipsum}
\usepackage{mdwlist}
\usepackage{textcomp}
\usepackage{multirow}
\usepackage{amsmath}
\usepackage{textcomp}
\usepackage{units}
\usepackage{chemfig}

\hypersetup{
    colorlinks,
    citecolor=black,
    filecolor=black,
    linkcolor=black,
    urlcolor=black
}

\makeatletter
\newcommand{\Rmnum}[1]{\expandafter\@slowromancap\romannumeral #1@}
\makeatother

\title{Redoxreaktionen: Explosionsbereich}
\author{Roland Rytz \and Niklaus Hofer}
\date{\today{}}

% Make title and author accessible in the header/footer
\makeatletter
  \let\Title\@title
  \let\Author\@author
\makeatother

\pagestyle{fancy}

%\geometry{a4paper, top=20mm, right=20mm, bottom=20mm, left=20mm}

\fancyhf{}      %delete default values
\setlength{\headwidth}{\textwidth}      %header and footer width equal the text width
\lhead{\Author}
\rhead{\Title}
\fancyfoot[CE,CO]{Speicherdatum: \today{}}
\fancyfoot[RE,RO]{\thepage}

\begin{document}
\maketitle
\newpage
\tableofcontents
\section{Hexan}
\subsection{Berechnungen}
Alle Angaben bei \unit[20]{°C}

Als Fl\"ussigkeit haben wir Hexan gew\"ahlt. Das verwendete Explosionsgef\"ass
hat ein Volumen von \(1L = 1dm^3\). Laut Tabelle ist Hexan z\"undbar
bei \(35-290 \frac{g}{m^3}\). Das entspricht \(0.035 - 0.290 \frac{g}{dm^3}\).

Um in einem sicheren Bereich zu liegen, haben wir den Mittelwert davon
gew\"ahlt. Das sind \(0.1625 \frac{g}{dm^3}\).
Hexan hat eine Dichte \(\rho\) von \(\rho = 0.659 \frac{g}{mL}\). Damit l\"asst
sich die f\"ur unser Experiment ben\"otigte Menge berechnen:
\[ \frac{0.1625g}{0.659\frac{g}{mL}} = 0.247mL \]

F\"ur den zweiten Versuch sollte eine Menge gew\"ahlt werden die ca. 50\% \"uber
dem oberen Explosionswert liegt. 
\[ \frac{1.5 \cdot 0.290g}{0.659\frac{g}{mL}} = 0.66mL \]

\subsection{Beobachtungen}
Den ersten Aufbau haben wir gleich nach dem Abf\"ullen des Hexans zu z\"unden
versucht. Dabei ist nat\"urlich nichts geschehen, da das Hexan zu diesem
Zeitpunkt noch nicht verdampft war und stattdessen als Fl\"ussigkeit auf dem
Grund des Gef\"asses lag. Es war also noch keine explosive oder z\"undbare
Mischung entstanden. Wir liessen das Hexangemisch deshalb stehen, w\"ahrend wir
den Versuch mit dem Erdgas vorbereiteten.

Nachdem wir den Versuch mit dem Erdgas vorbereitet hatten, war das Hexan
verdampft und wir versuchten erneut es zu z\"unden. Zuerst geschah wiederum
nichts. Uns fiel aber auf, dass es beim Z\"under keinen Funken gab. Wiederholtes
Versuchen (des Z\"undvorganges) f\"uhrte schlussendlich zum gew\"unschten
Ergebnis. Der Deckel des Gef\"asses wurde abgehoben und eine Stichflamme stach
kurz aus dem offenen Ende des Gef\"asses. Dabei und kurz darauf lief das
Gef\"ass kurz an um dann wieder durchsichtig zu werden.

Die Menge Hexan die f\"ur den zweiten Versuch verwendet wurde war deutlich
gr\"osser als die des ersten Versuches. Dadurch dauerte auch das Verdunsten
l\"anger. Durch das Verteilen der Fl\"ussigkeit am Gef\"assrand konnten wir die
Oberfl\"ache vergr\"ossern und somit die ben\"otigte Zeit zum Verdunsten
verringern. Wie erwartet geschah beim Z\"unden des Gemisches nichts. Wir
entfernten daraufhin den Deckel von dem Gef\"ass und richteten es schr\"ag nach
unten, so dass das Gas, das schwerer ist als die Luft, entweichen konnten. Dazu
bet\"atigten wir wiederholt den Z\"undknopf. Nach einigen Versuchen z\"undete
das Gemisch und es entstand eine Stichflamme die diejenige des ersten Versuches
in der Gr\"osse deutlich \"ubertraf.
\subsection{Fehlerabsch\"atzung}
Abgemessen haben wir das Hexan mit einer Eppendorfpipette.
\begin{table}[h!]
  \centering
  \begin{tabular}{|l|l|l|}\hline
    \bf Menge	& \bf Ungenauigkeit	& \bf Menge effektiv	\\
    \hline
    0.247mL	& \(\pm 1\%\)	&	0.24453 - 0.24947mL	\\
    \hline
    0.66mL	& \(\pm 1\%\)	&	0.6534 - 0.666mL	\\
    \hline
  \end{tabular}
  \caption{Fehlerabsch\"atzung bei den Versuchen mit Hexan}
  \label{tab:}
\end{table}
Die Messungenauigkeiten liegen also deutlich innerhalb des Explosionsbereiches
und k\"onnen deshalb vernachl\"assigt werden.

Ausserdem ist die genaue Zusammensetzung der Luft in dem Gef\"ass unbekannt.
\subsection{Interpretation}
Beim ersten Z\"undversuch, als das Hexan noch nicht verdampft war, hatte das
Gemisch die zur Z\"undung ben\"otigte Menge an Hexan in Gasform noch nicht
erreicht gehabt. Erst als alles Hexan verdampft war, lag es innerhalb des
Z\"undbereiches.

Mit der 1.5 Fachen Menge des oberen Randes des Z\"undbereiches war das Gemisch
wiederum in einem Zustand in dem keine Z\"undung statt finden konnte. Diesmal
allerdings nicht mangels Hexan, sondern mangels Sauerstoff. Nach dem Entfernen
des Deckels ist zunehmend Hexan entwichen und durch Sauerstoffhaltige Luft
ersetzt worden bis eine Z\"undung m\"oglich wurde.

\section{Erdgas}
\subsection{Berechnungen}
Alle Angaben bei \unit[20]{°C}

Erdgas hat einen Z\"undbereich von 5-15\% des Volumens. Auch hier haben wir
wieder den Mittelwert, also 10\%, genommen um im sicheren Bereich zu liegen. Das
Z\"undgef\"ass ist ein Typengleiches wie dasjenige welches wir f\"ur die
Versuche mit Hexan auch verwendet haben und hat ebenfalls \(1L = 1000mL\)
Volumen.
\[ 0.1 \cdot 1000mL = 100mL \]

F\"ur den zweiten Versuch, bei das 1.5 Fache des oberen Wertes des
Z\"undbereiches gew\"ahlt werden soll sehen die Berechnungen wie folgt aus:
\[ 1.5 \cdot 25\% = 22.5\% = 0.225 \]
\[ 0.225 \cdot 1000mL = 225mL \]
\subsection{Beobachtungen}
Das Abf\"ullen des Erdgases in das Explosiongef\"ass war nicht ganz einfach und
wurde dadurch erschwert, dass das Gas nicht sichtbar ist. Da Erdgas eine
geringere Dichte hat als die Luft und deshalb steigt, mussten wir das Gef\"ass
mit der \"Offnung nach unten halten und von unten her abf\"ullen. Bei den ersten
zwei Versuchen entfernten wir den Deckel vollst\"andig und f\"ullten das Gas
langsam von unten her in das Z\"undgef\"ass. Dabei ist aber ein betr\"achtlicher
Teil des Gases entwischt, was dazu gef\"uhrt hat, dass das Experiment
gescheitert ist.

Wir wurden von der Lehrkraft darauf aufmerksam gemacht, dass wir das Gas
abf\"ullen sollten, indem wir den Deckel nur am Rande leicht anheben und dann
das Gas einf\"ullen. So l\"asst sich der Deckel gleich nach dem Abf\"ullen
wieder verschliessen.

Beim dritten Versuch funktionierte es dann auch wie erwartet. Das Z\"undgef\"ass
war leicht schr\"ag nach oben gerichtet. Der Deckel wurde \glqq abgesprengt\grqq
und flog ca. zwei Meter weit. Dabei gab es eine deutlich vernehmbare
Ger\"auschentwicklung.

Bei dem zweiten Versuch, bei dem die deutlich gr\"ossere Menge Erdgas verwendet
wurde ist erwartungsgem\"ass nicht geschehen beim Z\"unden, obschon beim
Z\"under deutlich sichtbar Funken entstanden. Wir entfernten anschliessend den
Deckel und richteten die \"Offnung des Gef\"asses leicht schr\"ag nach oben (da
das Gas steigt) und dr\"uckten, w\"ahrend das Gas entwich, wiederholt den
Ausl\"oser. Nach einigen Versuchen entz\"undete sich das Gemisch und es entstand
eine Stichflamme die diejenige des ersten Versuches in ihrer Gr\"osse deutlich
\"ubertraf.
\subsection{Fehlerabsch\"atzung}
Das Erdgas haben wir mit einem Kolbenprober abgemessen. Das gestaltete sich
nicht ganz einfach, da die Zufuhr schwierig zu regeln war. Alleine dadurch ist
bereits eine ziemlich hohe Arbeitsungenauigkeit von ca. 4\% aufgetreten.
Gravierender aber noch, ist die Ugenauigkeit die durch das Abf\"ullen entstanden
ist. Wie bereits in den Beobachtungen erw\"ahnt hat das unvorsichtige Vorgehen
in einem Falle selsbt zum Scheitern des Versuches gef\"uhrt. Wir gehen hier
f\"ur die gegl\"uckten Experimente von einer Arbeitsungenauigkeit von ca. 10\%
aus.
\begin{table}[h!]
  \centering
  \begin{tabular}{|l|l|l|}\hline
    \bf Volumen in \%	& \bf Ungenauigkeit	& \bf Volumen in \% effektiv \\
    \hline
    10			& \(\pm 10\%\)		& 9.9 - 10.1 \\
    \hline
    22.5		& \(\pm 10\%\)		& 22.275 - 22.725 \\
    \hline
  \end{tabular}
  \caption{Fehlerabsch\"atzung bei den Versuchen mit Erdgas}
  \label{tab:}
\end{table}
Auch hier liegt also die Ungenauigkeit deutlich innerhalb des Z\"undbereiches
und kann deshalb vernachl\"assigt werden.

Ausserdem ist die genaue Zusammensetzung der Luft in dem Gef\"ass unbekannt.
\subsection{Interpretation}
Bei den ersten beiden Versuchen ist durch unser ungeschicktes Vorgehen beim
Abf\"ullen des Gases in das Z\"undgef\"ass zu viel Gas entwichen und es ist
keine z\"undbare Mischung entstanden da zu wenig Gas vorhanden war. Nachdem wir
das Gas vorsichtiger abgef\"ullt hatten, konnten nun eine Z\"undung stattfinden,
da die ben\"otigten Verh\"altnisse gegeben waren.

Mit der 1.5 Fachen Menge oberhalb des oberen Z\"undbereiches konnte das Gemisch
nicht gez\"undet werden, da zu wenig Sauerstoff vorhanden war. Als der Deckel
entfernt war, entwich immer mehr Gas und wurde durch sauerstoffhaltige Luft
ersetzt, solange bis ein z\"undbares Gemisch vorhanden war.

\section{Chemischer Hintergrund}
\subsection{Verbrennungsreaktion am Beispiel von Hexan}
Zusammensetzung von Hexan, siehe \cite{wiki:hexan}
\[zC_6H_{14} + nO_2 \to xCO_2 + yH_2O \]
\[2C_6H_{14} + 19O_2 \to 12CO_2 + 14H_2O \]
% UGLY HACK!!!
\newpage
\subsection{Oxidationszahlen}
\begin{figure}[h!]
  \centering
  \definesubmol{x1}{C^{-II}(-[:90]H^{I})(-[:270]H^{I})}
  \chemfig{C^{-III}(-[:90]H^{I})(-[:180]H^{I})(-[:270]H^{I})-!{x1}-!{x1}-!{x1}-!{x1}-C^{-III}(-[:0]H^{I})(-[:90]H^{I})(-[:270]H^{I})}
  %\chemfig{C(-[:o]H)(-[:90]H)(-[:270]H)(-[:180]C)}
  \caption{Oxidationszahlen von Hexan (\(C_6H_{14}\))}
  \label{fig:}
\end{figure}
\begin{figure}[h!]
  \centering
  \chemfig{\lewis{35,O}^0=\lewis{17,O}^0}
  \caption{Oxidationszahlen von Sauerstoff (\(O_2\))}
  \label{fig:}
\end{figure}
\begin{figure}[h!]
  \centering
  \chemfig{\lewis{35,O}^{-II}=C^{IV}=\lewis{17,O}^{-II}}
  \caption{Oxidationszahlen von Kohlenstoffdioxid (\(CO_2\))}
  \label{fig:}
\end{figure}
\begin{figure}[h!]
  \centering
  \chemfig{\lewis{24,O}^{-II}(-[:0]H^{I})(-[:270]H^{I})}
  \caption{Oxidationszahlen von Wasser (\(H_2O\))}
  \label{fig:}
\end{figure}
\subsection{Reduktion und Oxidation}
\(O^0 \to O^{-II} \) Reduktion (Zufuhr von 2 Elektronen) \\
\(C^{-II} \to C^{IV} \) Oxidation (Entzug von 6 Elektronen) \\
\(C^{-III} \to C^{IV} \) Oxidation (Entzug von 7 Elektronen) \\
\(H^I \to H^I \) Keine Ver\"anderung
\subsection{Redoxreaktion}
Oxidation 1: \( \color{red}2\color{black}C^{-\Rmnum{3}} -
\color{red}2\color{black}\cdot7e \to \color{red}2\color{black}C^{IV} \)\\
Oxidation 2: \( \color{red}4\color{black}C^{-\Rmnum{2}} -
\color{red}4\color{black}\cdot 6e \to \color{red}4\color{black}C^{\Rmnum{4} }
\)\\
Reduktion : \( \color{red}19\color{black}O^0 + \color{red}19\color{black}\cdot
2e \to \color{red}19\color{black}O^{-\Rmnum{2}} \)\\
Ausserdem : \( \color{red}14\color{black}H^I \to \color{red}14\color{black}H^I \)
\\
Redoxreaktion : \( 2C^{-III} + 4C^{-II} + 19O^0 + 14H^I \to 6C^{IV} + 19O^{-II}
+ 14H^I \)\\
Korrektur : \(\color{red}2\color{black}C_6H_{14} + \color{red}19\color{black}O_2
\to \color{red}12\color{black}CO_2 + \color{red}14\color{black}H_2O \)
\section{Berechnung der optimalen Menge}
Annahme: Die Luft in dem Gef\"ass besteht zu 100\% aus reinem Sauerstoff
(\(O_2\)). Ausserdem herrschen Normbedingungen.

1Mol Gas entspricht 22.4L. Das Gef\"ass hat also ein Volument von
\(\frac{1}{22.4}\) Mol. Die Anzahl Teilchen pro Mol ist festgelegt. In der
Verbrennungsgleichung sind 38 Sauerstoff (O) Teilchen, 12 Kohlenstoff (C)
Teilchen und 28 Wasserstoff (H) Teilchen enthalten.
Zusammen erhalten wir: \( 38 + 28 + 12 = 78 \) Teilchen.

Menge Sauerstoff: \( \frac{38}{78} \cdot \frac{1}{22.4} = 0.021749\) Mol \\
Menge Hexan:	  \( \frac{40}{78} \cdot \frac{1}{22.4} = 0.022894\) Mol

1 Mol Hexan l\"asst sich ermitteln durch Molzahlen von Kohlenstoff und
Wasserstoff: \(6 \cdot 12.011 + 14 \cdot 1.0079 = 86.1766 \frac{g}{mol} \)

\[ 0.022894 mol \cdot 86.1766\frac{g}{mol} = 1.97291g \]
\[ \frac{1.97291g}{0.659\frac{g}{mL}} = 2.99379mL \]

Die optimale Menge Hexan gem\"ass Verbrennungsgleichung w\"are demzufolge
2.99379mL.

Dieser Wert weicht nat\"urlich deutlich vom realen Wert ab, da die Luft nicht
nur aus Sauerstoff besteht, und die Raumtemperatur wiederholt nicht
ber\"ucksichtigt worden ist.
\section{Unterzeichnet}
Bern, \today

\vspace{0.5cm}
\noindent
Niklaus Hofer\hfill Roland Rytz

\vspace{2cm}
\noindent
\hrulefill \hfill \hrulefill

\newpage
\listoftables
\listoffigures
\bibliography{cus.bib}{}
\bibliographystyle{plain}
\end{document}

