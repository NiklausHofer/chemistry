\documentclass[11pt,paper=a4,final]{scrartcl}
\usepackage[utf8]{inputenc}
\usepackage{geometry}           %allows us to specify the 'seitenrand'
\usepackage{graphicx}           %package used to include graphics
\usepackage{hyperref}           %used to make klickable links
\usepackage{listings}
\usepackage{tabularx}
\usepackage{pdflscape}
\usepackage[figuresright]{rotating}
\usepackage{nameref}
\usepackage{longtable}
\usepackage{enumitem}
\usepackage{ngerman} % Make the document German
% Make the document German
\usepackage{ngerman}
\usepackage{fancyhdr}
\usepackage{lipsum}
\usepackage{mdwlist}
\usepackage{textcomp}
\usepackage{multirow}
\usepackage{amsmath}
\usepackage{textcomp}
\usepackage{units}
\usepackage{chemfig}

\hypersetup{
    colorlinks,
    citecolor=black,
    filecolor=black,
    linkcolor=black,
    urlcolor=black
}

\makeatletter
\newcommand{\Rmnum}[1]{\expandafter\@slowromancap\romannumeral #1@}
\makeatother

\title{Redoxreaktionen: Batterie (Daniell-Element)}
\author{Roland Rytz \and Niklaus Hofer}
\date{\today{}}

% Make title and author accessible in the header/footer
\makeatletter
  \let\Title\@title
  \let\Author\@author
\makeatother

\pagestyle{fancy}

%\geometry{a4paper, top=20mm, right=20mm, bottom=20mm, left=20mm}

\fancyhf{}      %delete default values
\setlength{\headwidth}{\textwidth}      %header and footer width equal the text width
\lhead{\Author}
\rhead{\Title}
\fancyfoot[CE,CO]{Speicherdatum: \today{}}
\fancyfoot[RE,RO]{\thepage}


\begin{document}
\maketitle
\newpage
\tableofcontents
\section{Messwerte}
\begin{table}[h!]
  \centering
  \begin{tabular}{|l|l|l|}\hline
    \bf Messung		& \bf Strom (I)	& \bf Spannung (U)	\\
    \hline
    5cm			& -		& 1.07V	\\
    \hline
    Voll		& -		& 1.08V \\
    \hline
    Motor		& 0.001A / 0.013A	& 0.98V / 0.40V \\
    \hline
    Lampe		& 0A		& 1.09V \\
    \hline
    \(5 \Omega\) Widerstand & 0.019A	& 0.17V \\
    \hline
    \(100 \Omega\) Widerstand & 0.006A	& 0.77V \\
    \hline
  \end{tabular}
  \caption{Messwerte}
  \label{tab:}
\end{table}
\section{Beobachtungen}
Wie im Aufbau beschrieben haben wir die beiden Beh\"alter zuerst bis auf eine
H\"ohe von ca. 5cm aufgef\"ullt. Die Messung an diesem Punkt ergab einen Wert
f\"ur die Spannung von \(U = 1.07V\). W\"ahrend des Nachf\"ullens der Elemente
haben wir das Messger\"at angeschlossen gelassen, dabei ist der Wert um
lediglich 0.1 Volt gestiegen. Beim aktivieren des Motors hat dieser nicht zu
drehen begonnen, ebenso hat die Lampe beim Zuschalten nicht geleuchtet. Aus
diesem Grund haben wir das Messger\"at ausgetauscht. Beim zweiten Messger\"at
sind dieselben St\"orungen aufgetreten. Ausserdem hat das zweite Messger\"at
Werte angezeigt die von denen des ersten Ger\"ates sehr verschieden waren. Dies
ist deutlich zu erkennen in den beim Motor angegebenen Werten.
\section{Fehlerabsch\"atzung}
Leider ist mir die Messtoleranz des verwendeten Messger\"ates nicht bekannt. Das
Ger\"at ist mit \glqq heliocentris DBGM Nr. 299.12.681.1\grqq beschriftet. Die
beim Wechsel des Messger\"ates festgestellte Ungenauigkeit betr\"agt aber \"uber
100\%! Jegliche andere Ungenauigkeit kann also getrost vernachl\"assigt werden.
\[\frac{100}{0.4} \cdot 0.98 -100 = 245 -100 = 145\]
Ich gehe also von einer Messungenauigkeit von \(\pm 145\%\) aus. Im negativen
Bereich heisst das, dass die Werte jeweils auch bei 0 liegen k\"onnten. Deshalb
enth\"alt die untere Tabelle in der hintersten Spalte jeweils den oberen
m\"oglichen Wert.
\begin{table}[h!]
  \centering
  \begin{tabular}{|l|l|l|l|l|}\hline
    \bf Messung		& \bf I	& \bf I (Ungenauigkeit) &\bf U & \bf U (Ungenauigkeit)	\\
    \hline
    5cm			& -		& - & 1.07V & 2.62	\\
    \hline
    Voll		& -& -		& 1.08V & 2.65\\
    \hline
    Motor		& 0.001A &  0.013A	& 0.40V & 0.98V \\
    \hline
    Lampe		& 0A	 & 0A       & 1.09V & 2.67\\
    \hline
    \(5 \Omega\) Widerstand & 0.019A& 0.0466A	& 0.17V  &0.417V\\
    \hline
    \(100 \Omega\) Widerstand & 0.006A	& 0.0147A& 0.77V & 1.887V\\
    \hline
  \end{tabular}
  \caption{Fehlerabsch\"atzung}
  \label{tab:}
\end{table}
Da Messwerte mit einer derart hohen Ungenauigkeit genau so gut als ung\"ultig
erkl\"art werden k\"onnen, werden wir die Ungenauigkeit im weiteren Verlauf des
Dokumentes nicht mehr beachten.
\section{Theoriewerte}
Die Reaktion findet statt zwischen dem Zink (Zn) und dem Kupfer (Cu). Gem\"ass
der Tabelle haben diese einen Wert von -0.76V (Zn) und +0.35V (Cu). Zusammen
gibt das einen theoretischen Spannungswert von 1.11V.

Verglichen mit dem h\"oheren unserer beiden Messwerte, 0.98V, ist die Abweichung
zimlich gering. Diese Abweichung kann verschieden Ursachen haben. Die
wahrscheinlichste ist in diesem Falle wohl die Ungenauigkeit des Messger\"ates.
Hinzu kommt aber noch der Umstand, dass der theoretische Wert von 1.11V nur bei
Normbedingungen gegeben ist, die im Labor nicht geherscht haben. Es ist
ausserdem auch vorstellbar, dass die verwendeten Metalle eine begrenzte
Leitf\"ahigkeit aufweisen und dadurch das Ergebnis weiter beeintr\"achtigen.
Bern, \today
\section{Widerstand}
Die Formel zum Berechnen des Widerstandes lautet\cite{wiki:ohm}
\[R = \frac{U}{I} \]
\begin{table}[h!]
  \centering
  \begin{tabular}{|l|l|l|l|}\hline
    \bf Messung		& \bf Strom (I)	& \bf Spannung (U) & \bf Widerstand (R)     \\
    \hline
    Motor		& 0.001A	& 0.98V		& \(980 \Omega\)\\
    \hline
    Lampe		& 0A		& 1.09V 	& - \\
    \hline
    \(5 \Omega\) Widerstand & 0.019A	& 0.17V 	& \(8.95 \Omega\) \\
    \hline
    \(100 \Omega\) Widerstand & 0.006A	& 0.77V 	& \(127.3 \Omega \)\\
    \hline
  \end{tabular}
  \caption{Widerstandswerte}
  \label{tab:}
\end{table}

Die Tabelle zeigt, dass der Motor einen sehr hohen Widerstand hat. Das ist gut
vorstellbar, da er ja auch tats\"achlich eine Arbeit verrichtet. Sie zeigt aber
auch, dass die Messwerte einen vom theoretischen Wert (des Messger\"ates)
abweichenden Wert ergeben. Die Werte f\"ur den \(5 \Omega\) und den \(100
\Omega\) Widerstand weichen deutlich ab. Auch hier k\"onnte der Eigenwiderstand
der Metalle eine Rolle spielen.

Mit dem wechselnden Widerstand lassen sich auch die Schwankungen in der Spannung
und Stromst\"arke erkl\"aren. Gem\"ass dem Ohmschen Gesetz m\"ussen die Werte
zusammenhangen. Wenn einer sich \"andert so muss das Auswirkungen auf einen oder
beide der anderen Werte haben.
\section{Interpretation des steigenden Widerstandes}
Beim Nachf\"ullen der Sulfate ist der gemessenen Widerstand um 0.01 Volt
gestiegen. Die Ursache f\"ur dieses Verhalten ist m\"oglicherweise, dass mit
zunehmendem Pegelstand der Fl\"ussigkeiten die Ber\"uhrungsfl\"ache mit dem
Metall steigt und dadurch mehr Elektronen simultan durch den Leiter wandern.
\section{Redoxgleichung}
Im Daniell-Element l\"auft folgende Reaktion ab:\\
Reduktion : \(Zn \to Zn^{\Rmnum{2}} + 2e^{-} \)\\
Oxidation: \(Cu^{\Rmnum{2}} + 2e^{-} \to Cu^0 \)\\
Redoxreaktion : \(Zn + Cu^{\Rmnum{2}} \to Zn^{\Rmnum{2}} +  Cu^0 \)\\

\section{Unterzeichnet}
\vspace{0.5cm}
\noindent
Niklaus Hofer\hfill Roland Rytz

\vspace{2cm}
\noindent
\hrulefill \hfill \hrulefill

\newpage
\listoftables
\listoffigures
\bibliography{battery.bib}{}
\bibliographystyle{plain}
\end{document}

