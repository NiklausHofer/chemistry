%----------------------------------------------------------------------------------------
%	PACKAGES AND OTHER DOCUMENT CONFIGURATIONS
%----------------------------------------------------------------------------------------

\documentclass[a4paper]{article}

\usepackage{lipsum} 

\usepackage[sc]{mathpazo} % Use the Palatino font
\usepackage[T1]{fontenc} % Use 8-bit encoding that has 256 glyphs
\linespread{1.05} % Line spacing - Palatino needs more space between lines
\usepackage{microtype} % Slightly tweak font spacing for aesthetics
\usepackage[hmarginratio=1:1,top=32mm,columnsep=20pt]{geometry} % Document margins
\usepackage{hyperref} % For hyperlinks in the PDF
\usepackage[hang, small,labelfont=bf,up,textfont=it,up]{caption} % Custom captions under/above floats in tables or figures
\usepackage{booktabs} % Horizontal rules in tables
\usepackage{float} % Required for tables and figures in the multi-column environment - they need to be placed in specific locations with the [H] (e.g. \begin{table}[H])
\usepackage{lettrine} % The lettrine is the first enlarged letter at the beginning of the text
\usepackage{paralist} % Used for the compactitem environment which makes bullet points with less space between them
\usepackage{ifthen}
\usepackage{ngerman}
\usepackage[utf8]{inputenc}


\usepackage{abstract}
\renewcommand{\abstractnamefont}{\normalfont\bfseries}
\renewcommand{\abstracttextfont}{\normalfont\itshape}

\usepackage{titlesec}
\titleformat{\section}[block]{\large\scshape\centering{\Roman{section}.}}{}{1em}{} % Change the look of the section titles 

\usepackage{lastpage}

\usepackage{fancyhdr}
\pagestyle{fancy}
\fancyhead{}
\fancyfoot{}
\fancyhead[R]{Versuchsprotokoll Zuckergehalt}
\fancyfoot[C]{\ifthenelse{\thepage=1}{}{ \thepage{} von \pageref{LastPage} }}

\hypersetup{
    colorlinks,
    citecolor=black,
    filecolor=black,
    linkcolor=black,
    urlcolor=black
}

%----------------------------------------------------------------------------------------
%	TITLE SECTION
%----------------------------------------------------------------------------------------

\title{
	\vspace{10mm}
	\fontsize{24pt}{10pt}
	\selectfont\textbf{
		Ermitteln des Zuckergehalts von Flüssigkeiten
	}\\[20mm]
}

\author{
	\Large{\textsc{Autoren}} \\[2mm] Niklaus \textsc{Hofer} \\[2mm] Roland \textsc{Rytz}
	\and
	\Large{\textsc{Versuchsaufseher}} \\[2mm]  Markus \textsc{Isenschmid} \\[120mm]
}
\date{Versuch durchgeführt am 27. August 2012 \\[2mm]
Bericht gespeichert am \today}


%----------------------------------------------------------------------------------------

\begin{document}

\maketitle
\newpage

\thispagestyle{fancy} % All pages have headers and footers

%----------------------------------------------------------------------------------------
%	ABSTRACT
%----------------------------------------------------------------------------------------

\begin{abstract}
\vspace{-2mm}
\noindent{Ziel des Versuchs war, den Zuckergehalt von verschiedenen Getränken (Rivella$\textsuperscript{\textregistered}$, Red Bull$\textsuperscript{\textregistered}$, Apfelschorle) sowie einer Zuckerlösung zu bestimmen. Dazu wurde die Dichte der Flüssigkeiten bestimmt und mit einem Refraktometer der Brechungsindex ermittelt. Es stellte sich heraus, dass unter den Getränken Red Bull$\textsuperscript{\textregistered}$ den höchsten Zuckergehalt aufweist.
}

\end{abstract}

%----------------------------------------------------------------------------------------
%	ARTICLE CONTENTS
%----------------------------------------------------------------------------------------

%\begin{multicols}{2} % Two-column layout throughout the main article text

\section{Vorgehensweise}
\lettrine[findent=0.25em,nindent=-0em,slope=0mm,lines=2]{D}  a Zucker eine höhere Dichte aufweist als Wasser und Süssgetränke überwiegend aus Wasser und Zucker bestehen, kann aus der Dichte des Getränks dessen ungefährer Zuckergehalt abgeleitet werden. Um die Dichte zu messen wurde die Flüssigkeit zuerst kräftig geschüttelt, da in den Getränken gelöste Kohlensäure die Resultate vermutlich verfälschen würde. Danach wurde mit einer Vollpipette eine Menge von 5ml abgemessen und in einem Glasgefäss gewogen. Jede Flüssigkeit wurde mindestens drei mal abgemessen und gewogen, um Ungenauigkeiten beim Abmessen mit der Vollpipette auszugleichen.\\
Nebst der Dichte wurde auch der Brechungsindex bestimmt. Dazu wurden von der bereits geschüttelten Flüssigkeit einige Tropfen auf das Glas eines Handrefraktometers gegeben und das Gerät wurde fokussiert.

%------------------------------------------------

\section{Berechnungen, Resultate}

Formel für die Dichte der Flüssigkeit:
\[sup\]

\begin{table}[H]
\caption{Rivella$\textsuperscript{\textregistered}$}
\centering
\begin{tabular}{lrrr}
\toprule
\bfseries Waage & Gewicht & Dichte & Zuckergehalt \\
\midrule
Messung 1 & $0$ & $0$ & $7.5$ \\
Messung 2 & $0$ & $0$ & $2$ \\
Messung 3 & $0$ & $0$ & $2$ \\
\midrule
$\bar{x}_{Waage}$ & $0$ & $0$ & $7.5 \pm 6\%$ \\
\toprule
\vspace{-1mm}
\bfseries Refraktometer & \multicolumn{2}{r}{Brechungsindex} & Zuckergehalt \\
\midrule
Messung 1 & & $0$ & $7.5$ \\
Messung 2 & & $0$ & $2$ \\
\midrule
$\bar{x}_{Refraktometer}$ & & $0$ & $7.5 \pm 6\%$ \\
\bottomrule
$\bar{x}_{Messwerte}$ & & & $6.4 \pm 7\%$ \\
\bottomrule
\end{tabular}
\end{table}

\lipsum[5] % Dummy text

\begin{equation}
\label{eq:emc}
e = mc^2
\end{equation}

\lipsum[6] % Dummy text

%------------------------------------------------

\section{Discussion}

\lipsum[7-8] % Dummy text

%----------------------------------------------------------------------------------------

%\end{multicols}

\end{document}
