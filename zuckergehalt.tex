%----------------------------------------------------------------------------------------
%	PACKAGES AND OTHER DOCUMENT CONFIGURATIONS
%----------------------------------------------------------------------------------------

\documentclass[a4paper]{article}

\usepackage{lipsum} 

\usepackage[sc]{mathpazo} % Use the Palatino font
\usepackage[T1]{fontenc} % Use 8-bit encoding that has 256 glyphs
\linespread{1.05} % Line spacing - Palatino needs more space between lines
\usepackage{microtype} % Slightly tweak font spacing for aesthetics
\usepackage[hmarginratio=1:1,top=32mm,columnsep=20pt]{geometry} % Document margins
\usepackage{hyperref} % For hyperlinks in the PDF
\usepackage[hang, small,labelfont=bf,up,textfont=it,up]{caption} % Custom captions under/above floats in tables or figures
\usepackage{booktabs} % Horizontal rules in tables
\usepackage{float} % Required for tables and figures in the multi-column environment - they need to be placed in specific locations with the [H] (e.g. \begin{table}[H])
\usepackage{lettrine} % The lettrine is the first enlarged letter at the beginning of the text
\usepackage{paralist} % Used for the compactitem environment which makes bullet points with less space between them
\usepackage{ngerman}
\usepackage[utf8]{inputenc}
\usepackage{url}


\usepackage{abstract}
\renewcommand{\abstractnamefont}{\normalfont\bfseries}
\renewcommand{\abstracttextfont}{\normalfont\itshape}

\usepackage{titlesec}
\titleformat{\section}[block]{\large\scshape\centering{\Roman{section}.}}{}{1em}{} % Change the look of the section titles 
\titleformat{\subsection}[block]{\vspace{-2mm}\centering{\Roman{section}.\scshape\roman{subsection}}}{}{1em}{} % Change the look of the subsection titles 

\usepackage{lastpage}

\usepackage{fancyhdr}
\pagestyle{fancy}
\fancyhead{}
\fancyfoot{}
\fancyhead[L]{Niklaus \textsc{Hofer} \\ Roland \textsc{Rytz}}
\fancyhead[R]{Versuchsprotokoll \\ Zuckergehalt}
\fancyfoot[C]{ \thepage{} von \pageref{LastPage} }

\hypersetup{
    colorlinks,
    citecolor=black,
    filecolor=black,
    linkcolor=black,
    urlcolor=black
}

\newcommand{\superscript}[1]{\ensuremath{^{\textrm{#1}}}}
\newcommand{\subscript}[1]{\ensuremath{_{\textrm{#1}}}}

%----------------------------------------------------------------------------------------
%	TITLE SECTION
%----------------------------------------------------------------------------------------

\title{
	\vspace{10mm}
	\fontsize{24pt}{10pt}
	\selectfont\textbf{
		Ermitteln des Zuckergehalts von Flüssigkeiten
	}\\[20mm]
}

\author{
	\Large{\textsc{Autoren}} \\[2mm] Niklaus \textsc{Hofer} \\[2mm] Roland \textsc{Rytz}
	\and
	\Large{\textsc{Versuchsaufseher}} \\[2mm]  Markus \textsc{Isenschmid} \\[120mm]
}
\date{Versuch durchgeführt am 27. August 2012 \\[2mm]
Bericht gespeichert am \today}


%----------------------------------------------------------------------------------------

\begin{document}

\maketitle
\newpage

\thispagestyle{fancy} % All pages have headers and footers

%----------------------------------------------------------------------------------------
%	ABSTRACT
%----------------------------------------------------------------------------------------

\begin{abstract}
\vspace{-2mm}
\noindent{Ziel des Versuchs war, den Zuckergehalt von verschiedenen Getränken (Rivella$\textsuperscript{\textregistered}$, Red Bull$\textsuperscript{\textregistered}$, Apfelschorle) sowie einer Zuckerlösung zu bestimmen. Dazu wurde die Dichte der Flüssigkeiten bestimmt und mit einem Refraktometer der Brechungsindex ermittelt. Es stellte sich heraus, dass unter den Getränken Red Bull$\textsuperscript{\textregistered}$ den höchsten Zuckergehalt aufweist.
}

\end{abstract}

%----------------------------------------------------------------------------------------
%	ARTICLE CONTENTS
%----------------------------------------------------------------------------------------

%\begin{multicols}{2} % Two-column layout throughout the main article text

\section{Vorgehensweise}
\lettrine[findent=0.25em,nindent=-0em,slope=0mm,lines=2]{D}  a Zucker eine höhere Dichte aufweist als Wasser und Süssgetränke überwiegend aus Wasser und Zucker bestehen, kann aus der Dichte des Getränks dessen ungefährer Zuckergehalt abgeleitet werden. Um die Dichte zu messen wurde die Flüssigkeit zuerst kräftig geschüttelt, da in den Getränken gelöste Kohlensäure die Resultate vermutlich verfälschen würde. Danach wurde mit einer Vollpipette eine Menge von 5ml abgemessen und in einem Glasgefäss gewogen. Jede Flüssigkeit wurde mindestens drei mal abgemessen und gewogen, um Ungenauigkeiten beim Abmessen mit der Vollpipette auszugleichen.\\
Nebst der Dichte wurde auch der Brechungsindex bestimmt, woraus ebenfalls auf den Zuckergehalt geschlossen werden kann. Dazu wurden von der bereits geschüttelten Flüssigkeit einige Tropfen auf das Glas eines Handrefraktometers gegeben. Danach wurde die Klappe geschlossen und das Gerät fokussiert.

%------------------------------------------------

\section{Messwerte, Beobachtungen}

\subsection{Messwerte}

%------------------- Red Bull

\begin{table}[H]
\caption{Bestimmung der Masse bei Red Bull\superscript{\textregistered}}
\centering
\begin{tabular}{lrr}
\toprule
Messung & Volumen \([ml]\) & Masse \([g]\)\\
\midrule
Messung 1 & \(5ml\) & \(5.210g\) \\
Messung 2 & \(5ml\) & \(5.243g\) \\
Messung 3 & \(5ml\) & \(5.225g\) \\
Messung 4 & \(5ml\) & \(5.233g\) \\
\midrule
$\bar{x}_{Messungen}$ & \(5ml\) & \(5.228g \pm 0.32\%\) \\
\bottomrule
\end{tabular}
\end{table}

\begin{table}[H]
\caption{Bestimmung des Brechungsindex bei Red Bull\superscript{\textregistered}}
\centering
\begin{tabular}{lr}
\toprule
Messung & Brechungsindex \\
\midrule
Messung 1 & \(1.349\) \\
Messung 2 & \(1.349\) \\
Messung 3 & \(1.349\) \\
\midrule
$\bar{x}_{Messungen}$ & \(1.349\) \\
\bottomrule
\end{tabular}
\end{table}

%------------------- Apfelschorle

\begin{table}[H]
\caption{Bestimmung der Masse bei Apfelschorle}
\centering
\begin{tabular}{lrr}
\toprule
Messung & Volumen \([ml]\) & Masse \([g]\)\\
\midrule
Messung 1 & \(5ml\) & \(5.140g\) \\
Messung 2 & \(5ml\) & \(5.119g\) \\
Messung 3 & \(5ml\) & \(5.141g\) \\
Messung 4 & \(5ml\) & \(5.137g\) \\
\midrule
$\bar{x}_{Messungen}$ & \(5ml\) & \(5.134g \pm 0.21\%\) \\
\bottomrule
\end{tabular}
\end{table}

\begin{table}[H]
\caption{Bestimmung des Brechungsindex bei Apfelschorle}
\centering
\begin{tabular}{lr}
\toprule
Messung & Brechungsindex \\
\midrule
Messung 1 & \(1.343\) \\
Messung 2 & \(1.343\) \\
Messung 3 & \(1.343\) \\
\midrule
$\bar{x}_{Messungen}$ & \(1.343\) \\
\bottomrule
\end{tabular}
\end{table}

%------------------- Rivella

\begin{table}[H]
\caption{Bestimmung der Masse bei Rivella\superscript{\textregistered}}
\centering
\begin{tabular}{lrr}
\toprule
Messung & Volumen \([ml]\) & Masse \([g]\)\\
\midrule
Messung 1 & \(5ml\) & \(5.172g\) \\
Messung 2 & \(5ml\) & \(5.135g\) \\
Messung 3 & \(5ml\) & \(5.138g\) \\
Messung 4 & \(5ml\) & \(5.165g\) \\
\midrule
$\bar{x}_{Messungen}$ & \(5ml\) & \(5.153g \pm 0.36\%\) \\
\bottomrule
\end{tabular}
\end{table}

\begin{table}[H]
\caption{Bestimmung des Brechungsindex bei Rivella\superscript{\textregistered}}
\centering
\begin{tabular}{lr}
\toprule
Messung & Brechungsindex \\
\midrule
Messung 1 & \(1.465\) \\
Messung 2 & \(1.465\) \\
Messung 3 & \(1.465\) \\
\midrule
$\bar{x}_{Messungen}$ & \(1.465\) \\
\bottomrule
\end{tabular}
\end{table}

%------------------- Testlösung

\begin{table}[H]
\caption{Bestimmung der Masse bei Testlösung A}
\centering
\begin{tabular}{lrr}
\toprule
Messung & Volumen \([ml]\) & Masse \([g]\)\\
\midrule
Messung 1 & \(5ml\) & \(5.400g\) \\
Messung 2 & \(5ml\) & \(5.438g\) \\
Messung 3 & \(5ml\) & \(5.389g\) \\
\midrule
$\bar{x}_{Messungen}$ & \(5ml\) & \(5.134g \pm 0.21\%\) \\
\bottomrule
\end{tabular}
\end{table}

\begin{table}[H]
\caption{Bestimmung des Brechungsindex bei Testlösung A}
\centering
\begin{tabular}{lr}
\toprule
Messung & Brechungsindex \\
\midrule
Messung 1 & \(1.367\) \\
Messung 2 & \(1.367\) \\
Messung 3 & \(1.367\) \\
\midrule
$\bar{x}_{Messungen}$ & \(1.367\) \\
\bottomrule
\end{tabular}
\end{table}

\vspace{5mm}

\subsection{Beobachtungen}

Die Getränke wirkten schon leicht abgestanden, Red Bull\superscript{\textregistered} hatte bereits einen faulen Geruch. Wie sich dies auf den Zuckergehalt auswirkt ist uns nicht bekannt.

%------------------------------------------------

\section{Berechnungen, Resultate}

Formel für die Dichte der Flüssigkeit (\(m\) betrug im Versuch stets \(5ml\)):
\[
	\rho = \frac{m}{V}
\]
\\
Formel für den Zuckergehalt der Flüssigkeit:
\[
	Zuckergehalt = \frac{(\rho-1)g}{cm^3}
\]
\\

\begin{table}[H]
\caption{Zuchergehalt nach Brechungsindex\superscript{\cite{eichtabelle}}}
\centering
\begin{tabular}{rr}
\toprule
\bfseries Brechungsindex & Zuckergehalt \\
\midrule
1.333	&	0\%		\\
1.3359	&	2\%		\\
1.3388	&	4\%		\\
1.3418	&	6\%		\\
1.3448	&	8\%		\\
1.3478	&	10\%	\\
1.3509	&	12\%	\\
1.3541	&	14\%	\\
1.3573	&	16\%	\\
1.3606	&	18\%	\\
1.3639	&	20\%	\\
1.3706	&	24\%	\\
1.3812	&	30\%	\\
1.3922	&	36\%	\\
1.4038	&	42\%	\\
1.4159	&	48\%	\\
1.433	&	56\%	\\
1.4511	&	64\%	\\
1.4654	&	70\%  \\
\bottomrule
\end{tabular}
\end{table}


Rivella\superscript{\textregistered} besteht laut Herstellerangabe zu 35\% aus Milchserum, was eine Dichte von \(\approx 1.034\frac{g}{cm^3}\) aufweist. \superscript{\cite{toepel}}

\begin{table}[H]
\caption{Rivella\superscript{\textregistered}}
\centering
\begin{tabular}{lrrr}
\toprule
\bfseries Waage & Gewicht & Dichte & Zuckergehalt \\
\midrule
Messung 1 & $0$ & $0$ & $7.5$ \\
Messung 2 & $0$ & $0$ & $2$ \\
Messung 3 & $0$ & $0$ & $2$ \\
\midrule
$\bar{x}_{Waage}$ & $0$ & $0$ & $7.5 \pm 6\%$ \\
\toprule
\vspace{-1mm}
\bfseries Refraktometer & \multicolumn{2}{r}{Brechungsindex} & Zuckergehalt \\
\midrule
Messung 1 & & $0$ & $7.5$ \\
Messung 2 & & $0$ & $2$ \\
\midrule
$\bar{x}_{Refraktometer}$ & & $0$ & $7.5 \pm 6\%$ \\
\bottomrule
$\bar{x}_{Messwerte}$ & & & $6.4 \pm 7\%$ \\
\bottomrule
Herstellerangabe & & & $6.4g$ \\
\bottomrule
\end{tabular}
\end{table}

%----------------------------------------------------------------------------------------

\section{Bibliographie}

{\def\section*#1{}
	\bibliographystyle{pccp}
	\bibliography{zuckergehalt}
	\nocite{*}
}

\end{document}
